\section*{Формальная постановка задачи}

Реализовать программное обеспечение для определения величины случайности заданной последовательности согласно собственному критерию.

Исходные данные:
\begin{itemize}
	\item последовательность из 10 1-значных случайных чисел, заданных пользователем;
	\item последовательность из 1000 1-, 2-, 3-значных случайных чисел, заданных алгоритмическим методом;
	\item последовательность из 1000 1-, 2-, 3-значных случайных чисел, заданных табличным методом.
\end{itemize}

\section*{Краткие теоритические сведения}

\subsection*{Алгоритмический метод}

В качестве алгоритмического метода генерации псевдослучайных чисел был реализован линейно конгруэнтный алгоритм. Каждый последующий член последовательности описывается с помощью рекурентной формулы:

\begin{equation}
	x_{x+1} = (a * x_n + c) \: mod \: m
\end{equation}

К выбору коэффциентов $a, c, m$ следует подходить с особой внимательностью с целью получения наиболее оптимальных значений. В рамках лабораторной работы были использованы случайные коэффициенты, сгенерированные стандартной библиотекой для оценки влияния коэффициентов на конечный результат.

\subsection*{Taбличный метод}

Для табличного метода были созданы специальные файлы с использованием стандартной библиотеки для генерации случайных чисел.

\subsection*{Критерий случайности последовательности чисел}

\textbf{Определение.} \textit{Собственной величиной схождения разности последовательности} будем называть необходимое количество повторений операций замены последовательности на последовательность, состоящую из модуля разности соседних элементов, из которой следует удалить подряд идущие элементы, пока не останется один элемент последовательности.

\textbf{Определение.} \textit{Величиной случайности последовательности} будем называть отношение собственной величины схождения разности последовательности к исходной длине последовательности.

Чем ближе величина случайности последовательности к единичному значению, тем лучше.


\section*{Средства реализации}

Язык программирования --- Python.

GUI --- QT.

\clearpage

\section*{Листинг кода}

\begin{lstlisting}
import random as r

def __gen_seed():
	m = r.randint(100000, 1000000)
	a = r.randint(10000, 100000)
	c = r.randint(10000, 100000)
	x0 = r.randint(10000, 100000)
	return m, a, c, x0


def gen(count: int, min: int, max: int) -> list[int]:
	m, a, c, x0 = __gen_seed()
	values = [x0]
	for _ in range(count):
		curr_value = (values[-1] * a + c) % m
		values.append(min + curr_value % (max - min + 1))
	return values[1:]

def leonov_check(arr: list[int]) -> float:
	diff_arr = []
	initial_len = len(arr)
	iters = 0
	while len(arr) != 1:
		for i in range(len(arr)-1):
			diff_arr.append(abs(arr[i+1] - arr[i]))
		
		new_arr = [diff_arr[0]]
		for i in range(1, len(diff_arr)):
			if diff_arr[i] != diff_arr[i-1]:
				new_arr.append(diff_arr[i])
		
		arr = new_arr
		diff_arr = []
		iters += 1
	
	return iters/(initial_len)
	

\end{lstlisting}




\clearpage

\section*{Демонстрация работы программы}

\boximg{70mm}{demo1}{Работа программы для возрастающей последовательности}

\boximg{70mm}{demo2}{Работа программы для убывающей последовательности}

\clearpage

\boximg{70mm}{demo3}{Работа программы для периодической последовательности}

\boximg{70mm}{demo4}{Работа программы для псевдослучайной последовательности}

\clearpage

\section*{Выводы}

Согласно рассмотренному критерию случайности последовательности линейно конгруентный алгоритм является менее случайным, чем используемые в стандартной библиотеке алгоритмы генерации последовательностей псевдослучайных чисел.

Следует отметить, что в разработанном критерие величина случайности последовательности зависит от разности между максимальным и минимальным значением в последовательности. Чем она больше для последовательности одной и той же длины, тем больше в среднем будет величина случайности последовательности.







