\section*{Формальная постановка задачи}


Слудует выполнить моделирование процесса выполнения заказов на автомобильном конвейере:
\begin{itemize}
	\item заказы от клиентов поступают в конструкторское бюро с интервалом $5\pm1$;
	\item существуют два вида заказов: гражданский и ведомственный, время утверждения которых составляет $10\pm2$ и $15\pm5$, время выполнения составляет $5\pm1$ и $8\pm3$ (требуется дополнительние дооснащение);
	\item существуют два оператора: регулярный и с более высоким уровнем допуска для обработки ведомственных заказов;
	\item при попадании заявки в бюро она передается оператору с наименьшим минимально необходимым уровнем допуска;
	\item после завершения процесса выпуска модели следует выполнить соответствующую сертификацию, время которой составляет $1\pm2$ и $3\pm2$ для гражданских и ведомственных заказов соответственно;
	\item по статистике $5\%$ гражданских и $2\%$ ведомственных автомобилей имеют дефекты, которые необходимо устранить, отправив автомобиль обратно в сборочное отделение на доработку, после которого будет выполнена повторная сертификация и отправка заказщику.
\end{itemize}

Будет выполнено моделирование плана на $300$ выпущенных автомобилей ($30\%$ заказов являются ведомственными). 

\boximg{24mm}{diagram}{Концептуальная схема}

\clearpage

Эндогенные переменные:
\begin{itemize}
	\item время обработки заказа каждым оператором;
	\item время выполнения каждого заказа;
	\item время прохождения сертификации;
	\item вероятность прохождения сертификации.
\end{itemize}

Экзогенные переменные:
\begin{itemize}
	\item количество выпущенных автомобилей каждого из типов;
	\item количество упущенных заказов на автомобили каждого из типов.
\end{itemize}



\section*{Средства реализации}

Язык программирования --- Python.

GUI --- QT.

\clearpage

\section*{Листинг кода}

\begin{lstlisting}
class Event:
	def __init__(self, time: float, stage: str, type: str = "None", operator: str = "None") -> None:
		self.time = time
		# STAGES = ["New", "Generated", "Designed", "Built", "Certifyed", "Finished"]
		self.stage = stage
		# TYPES = ["Default", "Government"]
		self.type = type
		# OPERATORS = ["Middle", "Senior"]
		self.operator = operator

	def new_stage(self, delta: float, new_stage: str):
		return Event(self.time + delta, new_stage, self.type, self.operator


class Events:
	def __init__(self) -> None:
		self.__pool: list[Event] = []
	
	def append(self, e: Event):
		i = 0
		while i < len(self.__pool) and self.__pool[i].time < e.time:
			i += 1
		if 0 < i < len(self.__pool):
			self.__pool.insert(i - 1, e)
		else:
			self.__pool.insert(i, e)
		
	def pop(self, index):
		return self.__pool.pop(index)
\end{lstlisting}

\clearpage

\begin{lstlisting}
class Model:
	def __init__(self,
				generator,
				operators,
				builders,
				certifiers) -> None:
		self.generator = generator
		self.operators = operators
		self.builders = builders
		self.certifiers = certifiers
		self.processed_tasks = {"Default": 0, "Government": 0}
		self.lost = {"Default": 0, "Government": 0}

	def total_processed(self):
		return self.processed_tasks["Default"] + self.processed_tasks["Government"]
	
	def start(self, total_tasks: int, g_rate: int, d_c_rate: int, g_c_rate: int):
		events = Events()
		events.append(Event(0, "New"))
		middle_flag = False
		senior_flag = False
		c_rates = {"Default": d_c_rate, "Government": g_c_rate}
		
		while self.total_processed() < total_tasks:
			e = events.pop(0)
			
			if e.stage == "New":
				if random.randint(0, 100) >= g_rate:
					e.type = "Default"
				else:
					e.type = "Government"
				new_e1 = e.new_stage(0, "Generated")
				events.append(new_e1)
				
				new_e2 = e.new_stage(self.generator.generate(),	"New")
				events.append(new_e2)
\end{lstlisting}

\clearpage

\begin{lstlisting}
			elif e.stage == "Generated":
				if (e.type == "Default" and middle_flag == False):
					e.operator = "Middle"
					middle_flag = True
				elif (e.type == "Default" and senior_flag == False) or \
					(e.type == "Government" and senior_flag == False):
					e.operator = "Senior"
					senior_flag = True
				else:
					self.lost[e.type] += 1
					continue
				new_e = e.new_stage(self.operators[e.type].generate(),"Designed")
				events.append(new_e)	
			elif e.stage == "Designed":
				if e.operator == "Middle":
					middle_flag = False
				elif e.operator == "Senior":
					senior_flag = False
				new_e = e.new_stage(self.builders[e.type].generate(), "Built")
				events.append(new_e)
			elif e.stage == "Built":
				new_e = e.new_stage(self.certifiers[e.type].generate(),	"Certifyed")
				events.append(new_e)
			elif e.stage == "Certifyed":
				if random.randint(0, 100) >= c_rates[e.type]:
					new_e = e.new_stage(0, "Finished")
				else:
					new_e = e.new_stage(self.builders[e.type].generate(), "Built")
				events.append(new_e)
			elif e.stage == "Finished":
				self.processed_tasks[e.type] += 1
			
		return self.processed_tasks, self.lost

	class BaseGenerator:
		def __init__(self, loc: float, scale: float = 0) -> None:
			self.loc = loc
			self.scale = scale
	
		def generate(self) -> float:
			return uniform.rvs(loc=self.loc - self.scale, scale=2 * self.scale, size=1)[0]
\end{lstlisting}


\clearpage

\section*{Демонстрация работы программы}

\boximg{125mm}{1}{Работа программы}
\boximg{125mm}{2}{Работа программы}
\clearpage







