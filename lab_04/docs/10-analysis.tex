\section*{Формальная постановка задачи}

Промоделировать систему с пошаговым принципом и событийным принципом, определив минимальный размер буферной памяти, при котором не будет потерь сообщений.

Исходные данные:
\begin{itemize}
	\item параметры генератора заявок;
	\item параметры обработчика заявок;
	\item приращение пошаговой модели;
	\item количество заявок;
	\item процент возврата заявок.
\end{itemize}

\section*{Краткие теоритические сведения}

\subsection*{Генератор}

Время прихода заявок из генератора описывается равномерным распределением.

\textbf{Равномерное распределение} --- распределение случайной величины, принимающей значения, принадлежащей некоторому промежутку конечной длины, характрезующееся тем, что плотность вероятности в этом промежутке всюду постоянна.

Функция распределения равномерной непрерывной случайной величины имеет следующий вид: 
\begin{equation*}
	F(x) = \begin{cases}
		0, & x \leq a \\
		\frac{x-a}{b-a}, & a \leq x \leq b \\
		1, & x > b
	\end{cases}
\end{equation*}

Плотность распределения равномерной непрерывной случайной величины имеет следующий вид:
\begin{equation*}
	f(x) = \begin{cases}
		\frac{1}{b-a}, & a \leq x \leq b \\
		0, & \text{иначе}
	\end{cases}
\end{equation*}

\subsection*{Обработчик}

Время обработки заявок из обработчика описывается распределением Эрланга.

\textbf{Распределение Эрланга} --- распределение, описывающее непрерывную случайную величину, принимающую неотрицательные значения и представляющую собой сумму $k$ независимых случайных величин, распределенных по одному и тому же экспоненциальному закону с параметром $\theta$.

Функция распределения Эрланга непрерывной случайной величины имеет следующий вид:
\begin{equation*}
	F(x) = 1 - e^{-x/\theta} \sum_{i=0}^{k-1} \frac{(x/theta)^i}{i!} 
\end{equation*}

Плотность распределения Эрланга непрерывной случайной величины имеет следующий вид:
\begin{equation*}
	f(x) = \frac{x^{k-1}e^{-x\theta}\theta^k}{(k-1)!}
\end{equation*}

\clearpage

\subsection*{Принципы моделирования работы системы}

\subsubsection{Пошаговый принцип}

Заключается в последовательном анализе состояний всех блоков системы в момент t + $\Delta t$ по заданному состоянию в момент t.  При этом новое состояние блоков определяется в соответствии с их алгоритмическим описанием с учетом действующих случайных факторов. В результате этого анализа принимается решение о том, какие системные события должны имитироваться на данный момент времени. Основным недостатком являются значительные затраты машинных ресурсов, а при недостаточном малых $\Delta t$ появляется опасность пропуска события. 

\subsubsection{Событийный принцип}

Состояние отдельных устройств изменяется в дискретные моменты времени, совпадающие в моментами поступления сообщения, окончания решения задачи, возникновения аварийных сигналов и т. д. При использовании событийного принципа состояния всех боков системы анализируется лишь в момент появления какого-либо события. Момент наступления следующего события определяется минимальным значением из списка будущих событий, представляющий собой совокупность моментов ближайшего изменения состояния каждого из блоков. Момент наступления следующего события определяется минимальным значением из списка событий.


\section*{Средства реализации}

Язык программирования --- Python.

GUI --- QT.

\clearpage

\section*{Листинг кода}

\begin{lstlisting}
import random

class TimeModel():
	def __init__(self, generator, processor, step: float) -> None:
		self.generator = generator
		self.processor = processor
		self.processed_tasks = 0
		self.step = step

	def start(self, total_tasks: int, repeat_rate: int) -> int:
		t_prev = 0
		t_current = self.step
		t_generator = self.generator.generate()
		t_processor = 0
		q_current_len = 0
		q_max_len = 0
		q_free = True

		while self.processed_tasks < total_tasks:
			if t_current > t_generator:
				q_current_len += 1
				if q_current_len > q_max_len:
					q_max_len = q_current_len
				t_prev = t_generator
				t_generator += self.generator.generate()
			if t_current > t_processor:
				if q_current_len > 0:
					q_prev = q_free
					if q_free:
						q_free = False
					else:
						self.processed_tasks += 1
						if random.randint(0, 100) <= repeat_rate:
							q_current_len += 1
					q_current_len -=1
					if q_prev:
						t_processor = t_prev + self.processor.generate()
					else:
						t_processor += self.processor.generate()
				else:
					q_free = True
			t_current += self.step
		return q_max_len
\end{lstlisting}

\clearpage

\begin{lstlisting}
import random 
	
class EventModel():
	def __init__(self, generator, processor) -> None:
		self.generator = generator
		self.processor = processor
		self.processed_tasks = 0

	def start(self, total_tasks: int, repeat_rate: int) -> int:
		events = Events()
		events.append(Event(self.generator.generate(), 'g'))
		q_current_len = 0
		q_max_len = 0
		q_free = True
		q_process = False

		while self.processed_tasks < total_tasks:
			e = events.pop(0)

			if e.type == 'g':
				q_current_len += 1
				if q_current_len > q_max_len:
					q_max_len = q_current_len
				new_e = Event(e.time + self.generator.generate(), 'g')
				events.append(new_e)
				if q_free:
					q_process = True

			elif e.type == 'p':
				self.processed_tasks += 1
				if random.randint(0, 100) <= repeat_rate:
					q_current_len += 1
				q_process = True

			if q_process:
				if q_current_len > 0:
					q_current_len -= 1
					new_e = Event(e.time + self.processor.generate(), 'p')
					events.append(new_e)
					q_free = False
				else:
					q_free = True
				q_process = False
			
		return q_max_len
\end{lstlisting}

\clearpage

\begin{lstlisting}
	class Event:
		def __init__(self, time: float, type: str) -> None:
			self.time = time
			self.type = type
	
	
	class Events:
		def __init__(self) -> None:
			self.__pool: list[Event] = []
		
		def append(self, e: Event):
			i = 0
			while i < len(self.__pool) and self.__pool[i].time < e.time:
				i += 1
			if 0 < i < len(self.__pool):
				self.__pool.insert(i - 1, e)
			else:
				self.__pool.insert(i, e)
			
		def pop(self, index):
			return self.__pool.pop(index)
\end{lstlisting}


\clearpage

\section*{Демонстрация работы программы}

\boximg{80mm}{1}{Время обработки больше времени прихода (возврат = 0\%)}
\boximg{80mm}{2}{Время обработки больше времени прихода (возврат = 25\%)}
\clearpage

\boximg{80mm}{3}{Время обработки больше времени прихода (возврат = 50\%)}
\boximg{80mm}{4}{Время обработки больше времени прихода (возврат = 75\%)}
\clearpage

\boximg{80mm}{5}{Время обработки больше времени прихода (возврат = 100\%)}
\boximg{80mm}{6}{Время обработки меньше времени прихода (возврат = 0\%)}
\clearpage

\boximg{80mm}{7}{Время обработки меньше времени прихода (возврат = 25\%)}
\boximg{80mm}{8}{Время обработки меньше времени прихода (возврат = 50\%)}
\clearpage

\boximg{80mm}{9}{Время обработки меньше времени прихода (возврат = 75\%)}
\boximg{80mm}{10}{Время обработки меньше времени прихода (возврат = 100\%)}






