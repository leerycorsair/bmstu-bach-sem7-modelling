\section*{Формальная постановка задачи}


Слудует выполнить моделирование процесса обслуживания клиентов в информационном центре:
\begin{itemize}
	\item клиенты приходят в интервалы времени $10\pm2$ мин;
	\item клиентов обслуживают три оператора, среднее время обработки заявки которых соответственно $20\pm5$, $40\pm10$, $40\pm20$ мин;
	\item клиент получает отказ, если все операторы заняты;
	\item клиент стремится занять оператора с наивысшей производительностью;
	\item полученные запросы сдаются в приемный накопитель, откуда они выбираются для обработки;
	\item первый компьютер получает запросы от первого и второго оператора, время обработки $15$ мин;
	\item второй компьютер получает запросы от третьего оператора, время обработки $30$ мин.
\end{itemize}

Для заданного количества заявок следует определить количество и процент потерянных запросов клиентов.

Эндогенные переменные: время обработки задания на $i-$ом операторе и время решения задания на $j-$ом компьютере.

Экзогенные переменные: число обслуженных клиентов и число клиентов, получивших отказ.

\section*{Средства реализации}

Язык программирования --- Python.

GUI --- QT.

\clearpage

\section*{Листинг кода}

\begin{lstlisting}
from model.base_generator import BaseGenerator

class RequestGenerator:
	def __init__(self, generator: BaseGenerator, cnt: int) -> None:
		self.generator = generator
		self.cnt = cnt
		self.receivers: RequestProcessor = []
		self.next = 0

	def generate(self) -> bool:
		self.cnt -= 1
		for receiver in self.receivers:
			if receiver.check_max():
				return receiver
		return None

	def delay(self) -> float:
		return self.generator.generate()

class RequestProcessor:
	def __init__(self, generator: BaseGenerator, max_queue_size: int = -1) -> None:
		self.generator = generator
		self.queue, self.received, self.max_queue, self.processed = 0, 0, max_queue_size, 0
		self.next = 0

	def check_max(self) -> bool:
		if self.max_queue == -1 or self.max_queue > self.queue:
			self.queue += 1
			self.received += 1
			return True
		return False

	def process(self) -> None:
		if self.queue > 0:
			self.queue -= 1
			self.processed += 1

	def delay(self) -> float:
		return self.generator.generate()
\end{lstlisting}

\clearpage

\begin{lstlisting}
from model.request import RequestGenerator, RequestProcessor

class Model:
	def __init__(self, generator: RequestGenerator, operators: list[RequestProcessor], computers: list[RequestProcessor]):
		self.generator = generator
		self.operators = operators
		self.computers = computers

	def event_mode(self):
		lost = 0
		generated_requests = self.generator.cnt
		self.generator.receivers = self.operators
		self.operators[0].receivers = [self.computers[0]]
		self.operators[1].receivers = [self.computers[0]]
		self.operators[2].receivers = [self.computers[1]]
		self.generator.next = self.generator.delay()
		self.operators[0].next = self.operators[0].delay()
		blocks = [self.generator,
			self.operators[0], self.operators[1], self.operators[2],
			self.computers[0], self.computers[1]]

		while self.generator.cnt >= 0:
			current_time = self.generator.next
			for block in blocks:
				if 0 < block.next < current_time:
					current_time = block.next

			for block in blocks:
			if current_time == block.next:
				if not isinstance(block, RequestProcessor):
					next_generator = self.generator.generate()
					if next_generator is not None:
						next_generator.next = current_time + next_generator.delay()
					else:
						lost += 1
					self.generator.next = current_time + self.generator.delay()
				else:
					block.process()
					if block.queue == 0:
						block.next = 0
					else:
						block.next = current_time + block.delay()
		return {"lost_percentage": lost / generated_requests * 100,
	"lost": lost}
\end{lstlisting}

\clearpage

\begin{lstlisting}
from scipy.stats import uniform

	class BaseGenerator:
		def __init__(self, loc: float, scale: float = 0) -> None:
			self.loc = loc
			self.scale = scale
	
		def generate(self) -> float:
			return uniform.rvs(loc=self.loc - self.scale, scale=2 * self.scale, size=1)[0]
\end{lstlisting}


\clearpage

\section*{Демонстрация работы программы}

\boximg{80mm}{1}{Работа программы}
\boximg{80mm}{2}{Работа программы}
\clearpage







