\section*{Формальная постановка задачи}

Реализовать программное обеспечение для построения графиков функции рапределения и фукнции плотности вероятности для случайных челичин, имеющих следующие распределения:
\begin{itemize}
	\item равномерное распределение;
	\item распределение Эрланга.
\end{itemize}

\section*{Краткие теоритические сведения}

\textbf{Равномерное распределение} --- распределение случайной величины, принимающей значения, принадлежащей некоторому промежутку конечной длины, характрезующееся тем, что плотность вероятности в этом промежутке всюду постоянна.

Функция распределения равномерной непрерывной случайной величины имеет следующий вид: 
\begin{equation*}
	F(x) = \begin{cases}
		0, & x \leq a \\
		\frac{x-a}{b-a}, & a \leq x \leq b \\
		1, & x > b
	\end{cases}
\end{equation*}

Плотность распределения равномерной непрерывной случайной величины имеет следующий вид:
\begin{equation*}
	f(x) = \begin{cases}
		\frac{1}{b-a}, & a \leq x \leq b \\
		0, & \text{иначе}
	\end{cases}
\end{equation*}

\textbf{Распределение Эрланга} --- распределение, описывающее непрерывную случайную величину, принимающую неотрицательные значения и представляющую собой сумму $k$ независимых случайных величин, распределенных по одному и тому же экспоненциальному закону с параметром $\theta$.

Функция распределения Эрланга непрерывной случайной величины имеет следующий вид:
\begin{equation*}
	F(x) = 1 - e^{-x/\theta} \sum_{i=0}^{k-1} \frac{(x/theta)^i}{i!} 
\end{equation*}

Плотность распределения Эрланга непрерывной случайной величины имеет следующий вид:
\begin{equation*}
	f(x) = \frac{x^{k-1}e^{-x\theta}\theta^k}{(k-1)!}
\end{equation*}

\section*{Средства реализации}

Язык программирования --- Python.

GUI --- QT и QTGraph.

\clearpage

\section*{Листинг кода}

\begin{lstlisting}
from math import exp, factorial
import numpy as np

def ud_cdf(x: float, a: float, b: float):
	if x < a:
		return 0
	elif x > b:
		return 1
	else:
		return (x - a) / (b - a)

def ud_pdf(x: float, a: float, b: float):
	if x >= a and x <= b:
		return 1 / (b - a)
	return 0

def erlang_cdf(x: float, k: float, theta: float):
	return 1 - exp(-x / theta) * sum((x/theta)**i/factorial(i) for i in range(k))

def erlang_pdf(x: float, k: float, theta: float):
	return x**(k-1) * exp(-x*theta) * theta**k / factorial(k-1)

def calc_ud_cdf(left: float, right: float, a: float, b: float):
	x = np.arange(left, right, (right-left)/1000)
	y = [ud_cdf(x, a, b) for x in x]
	return x, y

def calc_ud_pdf(left: float, right: float, a: float, b: float):
	x = np.arange(left, right, (right-left)/1000)
	y = [ud_pdf(x, a, b) for x in x]
	return x, y

def calc_erlang_cdf(max: float, k: int, theta: float):
	x = np.arange(0, max, max/1000)
	y = [erlang_cdf(x, k, theta) for x in x]
	return x, y

def calc_erlang_pdf(max: float, k: int, theta: float):
	x = np.arange(0, max, max/1000)
	y = [erlang_pdf(x, k, theta) for x in x]
	return x, y
\end{lstlisting}

\section*{Демонстрация работы программы}

\boximg{75mm}{ud1}{Равномерное распределение на отрезке $[-5,15]$ с параметрами $a=-5$ и $b=15$}

\boximg{75mm}{ud2}{Равномерное распределение на отрезке $[-7,19]$ с параметрами $a=-4$ и $b=11$}

\clearpage

\boximg{75mm}{erlang1}{Распределение Эрланга на отрезке $[0,30]$ с параметрами $k=1$ и $\theta=1$}

\boximg{75mm}{erlang2}{Распределение Эрланга на отрезке $[0,30]$ с параметрами $k=1$ и $\theta=5$}

\clearpage

\boximg{75mm}{erlang3}{Распределение Эрланга на отрезке $[0,30]$ с параметрами $k=10$ и $\theta=1$}







